%\documentclass[a4paper,11pt]{jsarticle}
\documentclass[a4paper,dvipdfmx]{jsarticle}
% ページ設定
\usepackage[top=2.5cm, bottom=2.5cm, left=2.5cm, right=2.5cm]{geometry}% 上下に2.5cm、左右に2.5cmの余白を取る
% 数式
\usepackage{amsmath,amsfonts}
\usepackage{bm}
% 画像
\usepackage[dvipdfmx]{graphicx}
\usepackage[dvipdfmx]{color}
% プログラム
\usepackage{listings,jlisting}
%プログラム挿入の設定
\lstset{
    basicstyle={\ttfamily},
    identifierstyle={\small},
    commentstyle={\smallitshape},
    keywordstyle={\small\bfseries},
    ndkeywordstyle={\small},
    stringstyle={\small\ttfamily},
    frame={tb},
    breaklines=true,
    columns=[l]{fullflexible},
    numbers=left,
    xrightmargin=0zw,
    xleftmargin=3zw,
    numberstyle={\scriptsize},
    stepnumber=1,
    numbersep=1zw,
    lineskip=-0.5ex
}
\renewcommand{\lstlistingname}{} % キャプション名の指定
% URL
\usepackage{url}
\usepackage[dvipdfmx]{hyperref}
\hypersetup{
    colorlinks=true,
    citecolor=blue,
    linkcolor=black,
    urlcolor=blue,
}
%pdfの挿入
\usepackage{pdfpages}



\begin{document}
%=====================================================================
\title{オペレーティングシステム中間課題}
\author{}
\date{\today}
\maketitle
%=====================================================================
\section{プログラムの説明}
    \subsection{作成したプログラム}
        哲学者の食事問題を解くプログラムを作成した.
    \subsection{クリティカルセクションと排他制御について行ったこと}
        今回作成したプログラムでは,フォークをリソースとして,哲学者がリソースを使用するという形になっている.
        哲学者がフォークを取得する部分をクリティカルセッションとしていて,優先で取得できるフォークがすでに取得されている場合は
        使用可能になるまで待機状態になる.\par
        本プログラムで排他制御を行わなかった場合,それぞれの哲学者は以下のように振る舞う.
        \begin{enumerate}
            \item 左のフォークが得られるまで思考し,フォークを取得する.
            \item 右のフォークが得られるまで思考し,フォークを取得する.
            \item 食事をする.(1~5秒のランダムな時間待機)
            \item 右のフォークを置く.
            \item 左のフォークを置く.
         \end{enumerate}
         この場合,全哲学者が左のフォークを持ち,右のフォークが得られなくなるのでデットロック状態になる.\par
         この問題の解決方法として,

%=====================================================================
\section{動作環境}
    \begin{table}[hbtp]
        \caption{動作環境}
        \label{table:env}
        \centering
        \begin{tabular}{|l|l|}
        \hline \hline
        OS  & ubuntu22.04 LTS\\\hline
        CPU  & Intel® Core™ i7-8700K CPU @ 3.70GHz × 12\\\hline
        メモリ  & 32.0GB \\\hline
        コンパイラ  &  g++ version 11.3.0\\\hline
        使用言語  &  C++17\\
        \hline \hline
        \end{tabular}
    \end{table}

%=====================================================================
\section{実行手順}
    プログラムのコンパイルコマンドを以下に示す.
    \begin{lstlisting}[caption=コンパイル,label=build]
        g++ -Wall main.cpp -pthread -std=c++17
    \end{lstlisting}\par
    以下のコマンドで実行した.
    \begin{lstlisting}[caption=実行コマンド,label=run]
        ./a.out
    \end{lstlisting}

%=====================================================================
\section{実行結果}
    ループ数3で哲学者が5人の条件でプログラムの実行したときの結果を以下に示す.
    \begin{lstlisting}[caption=実行結果,label=run_res]
        f11@desktop-f11<TUThumbleERv3>:~/Git/The_edining_philosophers$ ./a.out
        start
        Philosopher 1 is reading..
        Philosopher 2 is reading..
        Philosopher1 thinking
        Philosopher 3 is reading..
        Philosopher 4 is reading..
        Philosopher 5 is reading..
        Philosopher3 thinking
        Philosopher4 thinking
        PhilosopherPhilosopher5 thinking
        2 thinking
        Philosopher4 get fork4 DONE 
        Philosopher4 get fork3 DONE 
        Philosopher4 is eating
        Philosopher1 get fork5 DONE 
        Philosopher1 get fork1 DONE 
        Philosopher1 is eating
        Philosopher3 get fork2 DONE 
        Philosopher3 get fork3Philosopher4 has finished eating
        Philosopher4 put fork3 DONE 
        Philosopher4 put fork4 DONE 
        Philosopher4 thinking
        DONE 
        Philosopher3 is eating
        Philosopher5 get fork4 DONE 
        Philosopher5 get fork5Philosopher2 get fork2Philosopher3 has finished eating
        Philosopher3 put fork3 DONE 
        Philosopher3 put fork2 DONE 
        Philosopher3 thinking
        DONE 
        Philosopher2 get fork1Philosopher1 has finished eating
        Philosopher1 put fork1 DONE 
        Philosopher1 put fork5 DONE 
        Philosopher1 thinking
        DONE 
        Philosopher2 is eating
        DONE 
        Philosopher5 is eating
        Philosopher4 get fork4Philosopher3 get fork2Philosopher1 get fork5PhilosopherPhilosopher5 has finished eating
        Philosopher5 put fork5 DONE 
        Philosopher5 put fork4 DONE 
        Philosopher5 thinking
        DONE 
        Philosopher4 get fork3 DONE 
        Philosopher4 is eating
        DONE 
        Philosopher1 get fork12 has finished eating
        Philosopher2 put fork1 DONE 
        Philosopher2 put fork2 DONE 
        Philosopher2 thinking
        DONE  DONE 
        Philosopher1 is eating

        Philosopher3 get fork3Philosopher4 has finished eating
        Philosopher4 put fork3 DONE 
        Philosopher4 put fork4 DONE 
        Philosopher4 thinking
        DONE 
        Philosopher3 is eating
        Philosopher1 has finished eating
        Philosopher1 put fork1 DONE 
        Philosopher1 put fork5 DONE 
        Philosopher1 thinking
        Philosopher4 get fork4 DONE 
        Philosopher4 get fork3Philosopher5 get fork4Philosopher3 has finished eating
        Philosopher3 put fork3 DONE 
        Philosopher3 put fork2 DONE 
        Philosopher3 thinking
        Philosopher1 get fork5 DONE 
        Philosopher1 get fork1 DONE 
        Philosopher1 is eating
        DONE 
        Philosopher4 is eating
        Philosopher2 get fork2 DONE 
        Philosopher2 get fork1Philosopher4 has finished eating
        Philosopher4 put fork3 DONE 
        Philosopher4 put fork4 DONE 
        DONE 
        Philosopher5 get fork5Philosopher1 has finished eating
        Philosopher1 put fork1 DONE 
        Philosopher1 put fork5 DONE 
        DONE 
        Philosopher2 is eating
        DONE 
        Philosopher5 is eating
        Philosopher2 has finished eating
        Philosopher2 put fork1 DONE 
        Philosopher2 put fork2 DONE 
        Philosopher2 thinking
        Philosopher3 get fork2 DONE 
        Philosopher3 get fork3 DONE 
        Philosopher3 is eating
        Philosopher5 has finished eating
        Philosopher5 put fork5 DONE 
        Philosopher5 put fork4 DONE 
        Philosopher5 thinking
        Philosopher5 get fork4 DONE 
        Philosopher5 get fork5 DONE 
        Philosopher5 is eating
        Philosopher2 get fork2Philosopher3 has finished eating
        Philosopher3 put fork3 DONE 
        Philosopher3 put fork2 DONE 
        DONE 
        Philosopher2 get fork1 DONE 
        Philosopher2 is eating
        Philosopher2 has finished eating
        Philosopher2 put fork1 DONE 
        Philosopher2 put fork2 DONE 
        Philosopher5 has finished eating
        Philosopher5 put fork5 DONE 
        Philosopher5 put fork4 DONE 
        finish
    \end{lstlisting}

%=====================================================================
\section{工夫した点・感想}


%=====================================================================
\section{参考文献}
\begin{thebibliography}{99}
    %・Webページ
    \item Akira Takahashi ," std::thread",cpprefjp - C++日本語リファレンス,\url{https://cpprefjp.github.io/reference/thread/thread.html},2022年11月29日.

\end{thebibliography}
%=====================================================================



\end{document}